\section{Implementation} % (fold)
\label{sec:implementation}

\subsection{Search Methods} % (fold)
\label{sub:search_methods}
Each search method is a derived form of the base \mintinline[bgcolor=codebg]{cpp}{SearchMethod} class, implementing the \mintinline[bgcolor=codebg]{cpp}{search()} and \mintinline[bgcolor=codebg]{cpp}{get_child()} virtual methods.
% subsection search_methods (end)

\paragraph{BFS} % (fold)
\label{par:bfs}
BFS was implemented iteratively, using a FIFO frontier, its pseudocode shown below.

\begin{breakablealgorithm}
	\caption{BFS}
	\begin{algorithmic}[1]
		\Procedure{search}{$env$: a problem environment}
			\State $n \gets$ starting state, $explored \gets$ \{\}, $frontier \gets$ empty FIFO queue
			\If {$n = goal$} \Return solution \EndIf
			\State add $n$ to $frontier$
			\While{$frontier \neq empty$}
				\State $n \gets$ first node in frontier
				\ForAll{valid child $c$ with action $a$ in $env$}
					\If{not explored}
						\State add $c$ to explored set
						\If{$c=goal$} \Return solution \EndIf
						\State add to frontier
					\EndIf
				\EndFor
			\EndWhile
			\State \Return no solution found
		\EndProcedure
	\end{algorithmic}
\end{breakablealgorithm}
% paragraph bfs (end)

\paragraph{DFS} % (fold)
\label{par:dfs}
DFS is implemented in a very similar way to the pseudocode outlined above for BFS, however it uses a LIFO queue as it's frontier instead of a FIFO queue.
% paragraph dfs (end)

\paragraph{GBFS and A*} % (fold)
\label{par:gbfs_and_a_}
GBFS and A* are again both implemented similarily to BFS and DFS but using a a priority queue as their frontier. Furthermore, when getting a valid child from the environment, both add their $f$ cost to the child node before adding it to the frontier, to do this they override \mintinline[bgcolor=codebg]{cpp}{SearchMethods}'s \mintinline[bgcolor=codebg]{cpp}{get_child()} method. GBFS will simply assign as its $f$ cost the distance to the goal using one of the distance functions outlined in section~\ref{sub:as}, while A* will sum both the distance and the child nodes parent $f$ value to get $f(n)=g(n) + h(n)$.
% paragraph gbfs_and_a_ (end)

\paragraph{IDDFS} % (fold)
\label{par:iddfs}
IDDFS is implemented recursively in the following manner using a specialized \mintinline[bgcolor=codebg]{cpp}{IDDFSResults} struct wrapping a cutoff value and the solution if it exists in a single structure.

\begin{breakablealgorithm}
\caption{IDDFS implementation pseudocode}
\begin{algorithmic}[1]
	\Procedure{search}{$env$: a problem environment}
		\State $results \gets$ empty results set, $\ell \gets 1$
		\While{$\mathit{results.cutoff}$}
			\State $results \gets$ \Call{depth\_limited\_search}{$env,\ \ell$}
			\State $\ell \gets \ell + 1$
		\EndWhile
		\State \Return $results$
	\EndProcedure

	\State

	\Procedure{depth\_limited\_search}{$env$: a problem environment, $\ell$}
		\State $n \gets$ starting node in $env$
		\State \Return \Call{recursive\_dls}{$n,\ env,\ \ell$}
	\EndProcedure

	\State

	\Procedure{recursive\_dls}{$env$: problem environment, $n$: a node. $d$: the current depth}
		\State Add $n$ to explored set
		\If{$n=goal$} \Return solution \EndIf
		\If{$d < 1$} \Return cutoff \EndIf
		\State $\mathit{cutoff\_occurred} \gets false$, $results \gets$ empty results set
		\ForAll{children $c$ of $n$ with valid actions $a$ in $env$}
			\If{$c$ not in explored set}
				\State $results \gets$ \Call{recursive\_dls}{$c,\ env,\ d-1$}
				\If{$\mathit{results.cutoff}$}
					$\mathit{cutoff\_occurred} \gets true$
				\ElsIf{solution found in $results$}
					\Return $results$
				\EndIf
			\EndIf
		\EndFor
		\State \Return $\mathit{cutoff\_occurred}$
	\EndProcedure
\end{algorithmic}
\end{breakablealgorithm}
% paragraph iddfs (end)

\paragraph{IDA*} % (fold)
\label{par:ida_}
IDA* is implemented almost identically to to IDDFS, however it keeps a temporary map of the current path to ensure that nodes in the same path aren't repeated, alongside using the last encountered $f$ value higher than the current $\ell$ as the cutoff.
% paragraph ida_ (end)

\subsection{Frontier} % (fold)
\label{sub:frontier}
Each search method utilized a different frontier container type. This is provided via the \mintinline[bgcolor=codebg]{cpp}{Frontier<T>} template type which has FIFO (\mintinline[bgcolor=codebg]{cpp}{Frontier<std::queue>}), LIFO (\mintinline[bgcolor=codebg]{cpp}{Frontier<std::vector>}), and Priority Queue (\mintinline[bgcolor=codebg]{cpp}{Frontier<std::priority_queue>}) template specializations which implement their methods in a container-specific way whilst providing a common interface for search methods.
% subsection frontier (end)

\subsection{Explored Set} % (fold)
\label{sub:exploredset}
The \mintinline[bgcolor=codebg]{cpp}{ExploredSet} class provides an interface for adding and querying expanded nodes and denoting them as \textit{explored}. This is implemented internally using an \mintinline[bgcolor=codebg]{cpp}{std::unordered_map} and several helper functions with a high-level interface for dealing with nodes directly. The explored set also contains an \mintinline[bgcolor=codebg]{cpp}{std::vector} of all the operations that occurred during the search algorithm for use in the visualizer (see~\ref{sub:visualizer}) to use in displaying the search tree in order.
% subsection exploredset (end)

\subsection{Environment} % (fold)
\label{sub:environment}
Finally, the environment class defines valid actions and generates new child nodes based on the current \textit{state} - a point in the grid. It also defines a \textit{goal test} for determining whether a given state is the goal.
% subsection environment (end)

% section implementation (end)
